\documentclass[a4paper,12pt]{article}
\usepackage[utf8]{inputenc}
\usepackage[T1]{fontenc}
\usepackage{lmodern} % Tipografía moderna
\usepackage{newunicodechar}
\usepackage{geometry}
\geometry{margin=1in}
\usepackage{amsmath}
\usepackage{graphicx}
\usepackage{setspace}
\usepackage{titlesec}
\usepackage{listings}
\usepackage{ragged2e}
\usepackage{color}
\usepackage{xcolor}
\usepackage{fancyhdr}
\usepackage{tcolorbox}
\usepackage{hyperref}
\usepackage{tikz}

% Estilo de portada
\newtcolorbox{portadabox}[2][]{colback=white!5!gray, colframe=black,
title=#2, fonttitle=\bfseries, #1}

% Configuración de encabezados
\pagestyle{fancy}
\fancyhf{}
\fancyhead[L]{Proyecto Algoritmos: Juego}
\fancyhead[R]{2024}
\fancyfoot[C]{\thepage}

% Configuración de estilo de código
\lstset{
    language=Python,
    basicstyle=\ttfamily\small,
    keywordstyle=\color{blue}\bfseries,
    commentstyle=\color{green!50!black},
    stringstyle=\color{red},
    numberstyle=\tiny\color{gray},
    numbers=left,
    stepnumber=1,
    numbersep=8pt,
    backgroundcolor=\color{white},
    frame=single,
    breaklines=true,
    breakatwhitespace=true,
    captionpos=b,
    showspaces=false,
    showstringspaces=false,
    tabsize=4
}

% Estilo de títulos
\titleformat{\section}{\normalfont\Large\bfseries}{\thesection}{1em}{}
\titleformat{\subsection}{\normalfont\large\bfseries}{\thesubsection}{1em}{}
\titleformat{\subsubsection}{\normalfont\normalsize\bfseries}{\thesubsubsection}{1em}{}

% Divisores decorativos
\newcommand{\divider}{
    \begin{center}
        \tikz{\draw[thick, color=gray] (0,0) -- (15,0);}
    \end{center}
}

% Configuración de hipervínculos
\hypersetup{
    colorlinks=true,
    linkcolor=blue,
    filecolor=magenta,
    urlcolor=cyan,
    pdftitle={Calculadora de Nómina},
    pdfpagemode=FullScreen,
}

\begin{document}

% Portada
\begin{titlepage}
    \begin{center}
        \vspace*{1cm}
        
        % Borde decorativo
        \begin{portadabox}[width=\textwidth]{}
            \begin{center}
                \includegraphics[width=0.3\textwidth]{Logo_URosario (2).png} % Logo de la universidad
                \vspace{1cm}
                
                \Huge\textbf{Proyecto Algoritmos}\\[1cm]
                \Large\textbf{La gran Colombia}\\[2cm]
                \Large\textbf{Documentación}\\[2cm]
                \Large\textbf{2025}\\[2cm]
            \end{center}
        \end{portadabox}
    \end{center}
\end{titlepage}

% Roles de los integrantes
\section*{Roles de los Integrantes}
\begin{itemize}
    \item \textbf{Santiago Diaz}: Gerente.
    \item \textbf{Gabriela Morales}: Diseño de juego y Algoritmos. 
    \item \textbf{Danna Gonzales}: Desarrollador de backend y estructura de datos. 
    \item \textbf{Valentina Mesa}: Doumentación y pruebas. 
\end{itemize}

\divider

\tableofcontents
\newpage

% Introducción
\section{Introducción}

\textbf{“La Gran Colombia”} es un prototipo académico de juego de estrategia por turnos 1 vs 1, inspirado en Risk y Supremacy 1914, ambientado en el mapa histórico de la Gran Colombia (Colombia, Venezuela, Ecuador y Panamá, 1821-1831). El tablero inicial está compuesto por 12 provincias interconectadas, suficiente para mostrar mecánicas de conquista sin sobrecargar al jugador. En cada ronda, los jugadores rivales planifican movimientos y ataques; luego todas las acciones se resuelven simultáneamente antes de pasar al siguiente turno.

\begin{description}
    \item[Metas del prototipo v0.1] 
\end{description}
\begin{enumerate}
    \item \textbf{Mapa jugable} con las 12 provincias clicables y resaltado al pasar el cursor.
    \item \textbf{Sistema básico de tropas y recursos} dinero y unidades de infantería.
    \item \textbf{Condición de victoria clara} controlar  8 de las 12 provincias.
    \item \textbf{Interfaz minima} panel de información de provincia, botones “Mover” y “Atacar”, contador de turnos.
\end{enumerate}
% Requisitos
\section{Requisitos}
\begin{justify}
    Para ejecutar este proyecto localmente se necesita: 
\end{justify}
\begin{itemize}
    \item \textbf {Sistema operativo:} Windows 10/11, macOS 10.13+ o Linux (Ubuntu 18.04+)
    \item \textbf{Unity} (versión recomendada: 2021.3 LTS o superior)
    \item \textbf{Procesador:} Intel Core i3 o equivalente. 
    \item \textbf{Memoria RAM}: 4 GB mínimo. 
    \item \textbf{Espacio en disco}: 1 GB libre. 
    \item \textbf{Tarjeta gráfica}: Compatible con OpenGL 3.2 o superior. 
    \item \textbf{Paquete TextMeshPro} (se instala automáticamente al abrir el proyecto en Unity si falta). 
\end{itemize}
\subsection{Instalación}
\begin{enumerate}
    \item \textbf{Clona o descarga el repositorio:}
    \begin{lstlisting}[language=bash]
        git clone https://github.com/usuario/AlgoritmosProyecto.git
    \end{lstlisting}
    \item \textbf{Abre el proyecto en Unity:}
    \begin{itemize}
        \item Abre Unity Hub.
        \item Haz clic en "Add" y selecciona la carpeta del proyecto descargado.
    \end{itemize}
    \item \textbf{Instala dependencias si es necesario:}
    \begin{itemize}
        \item Al abrir el proyecto, Unity puede pedirte instalar TextMeshPro u otros paquetes. Acepta e instala.
    \end{itemize}
    \item \textbf{Ejecuta el juego}
    \begin{itemize}
        \item Selecciona la escena principal (por ejemplo, MenuInicial o la escena de inicio).
        \item Haz clic en el botón \textbf{Play} en el editor de Unity para probar el juego localmente.
    \end{itemize}
\end{enumerate}
\divider

% Manual de Usuario
\section{Manual de Usuario}

\begin{description}
    \item[Menú Principal]
\end{description}
\begin{itemize}
        \item Al iniciar el juego, verás el menú principal con las opciones Jugar y Salir.
        \item Pulsa Jugar para pasar a la selección de país.
        \item Pulsa Salir para cerrar el juego.
    \end{itemize}
\begin{description}
    \item[Selección de País]
\end{description}
 \begin{itemize}
        \item Elige el país para cada jugador usando los menús desplegables.
        \item No es posible seleccionar el mismo país para ambos jugadores.
        \item Pulsa \textbf{Confirmar} para iniciar la partida con las selecciones realizadas.
\end{itemize}
\begin{description}
    \item[Juego] 
\end{description}
\begin{itemize}
    \item El mapa muestra las provincias disponibles. Cada jugador controla un conjunto de provincias iniciales.
    \item El juego es por turnos. En cada turno, el jugador puede construir edificios, reclutar tropas y mover unidades.
    \item Accede al menú de construcción seleccionando una provincia. 
    \item Construye edificios si tienes recursos suficientes.
    \item Recluta tropas en provincias con los edificios adecuados.
    \item Ataca provincias adyacentes controladas por el enemigo.
    \item El resultado de la batalla depende de las tropas y edificios defensivos.
    \item El juego termina cuando un jugador controla todas las provincias o cumple condiciones de victoria específicas.
\end{itemize}
\divider

\section{Guía Interna del Código}

\subsection{Resumen de las funciones principales}
\begin{itemize}
    \item \textbf{Simulación de Batallas:} \colorbox{yellow!30}{\texttt{Ejercito.ResolverBatalla}} calcula el resultado de los combates entre ejércitos, considerando ataque, defensa y efectos de edificios.
    \item \textbf {Administración de recursos:} \colorbox{yellow!30}{\texttt{RecursosJugador}}gestiona los recursos del jugador, verifica pagos y produce recursos por turno.
    \item \textbf{Construcción y producción:} \colorbox{yellow!30}{\texttt{Departamento}} y \colorbox{yellow!30}{\texttt{Edificios}} gestionan la construcción de edificios y la producción de tropas y recursos.
    \item \textbf{Gestión de turnos: }\colorbox{yellow!30}{\texttt{SistemaDeTurnos}}controla el flujo de turnos y la producción automática de recursos y construcciones.
    \item \textbf{Menús y navegación:} Scripts como: 
    \begin{lstlisting}[language=c, caption={Clase Menú inicial}]
        using System.Collections;
        using System.Collections.Generic;
        using UnityEngine;
        using UnityEngine.SceneManagement; 
        public class MenuInicialScript : MonoBehaviour
        {
            public GameObject PanelMenuInicial;   // Panel con botones Jugar y Salir
            public GameObject SeleccionPaisPanel;   // Panel de seleccion de pais
        
            void Start()
            {
                PanelMenuInicial.SetActive(true);     // Mostrar menu principal
                SeleccionPaisPanel.SetActive(false);    // Ocultar seleccion pais
            }
        
            public void Jugar()
            {
                PanelMenuInicial.SetActive(false);    // Oculta el menu principal
                SeleccionPaisPanel.SetActive(true);
            }
        
            public void Salir()
            {
                Debug.Log("Saliendo...");
                Application.Quit();
            }
        
        }
    \end{lstlisting}

    \begin{lstlisting}[language=c, caption={SeleccionPaisController}]
    using System.Collections.Generic;
    using UnityEngine;
    using UnityEngine.UI;
    using TMPro;
        
    public class SeleccionPaisController : MonoBehaviour
    {
        public TMP_Dropdown dropdownJugador1;
        public TMP_Dropdown dropdownJugador2;
        public Button botonConfirmar;
        
        private List<string> opcionesOriginales = new List<string> { "Nueva Granada", "Ecuador", "Venezuela" };
        
        void Start()
        {
            // Carga las opciones en los dropdowns
            dropdownJugador1.ClearOptions();
            dropdownJugador1.AddOptions(opcionesOriginales);
        
            dropdownJugador2.ClearOptions();
            dropdownJugador2.AddOptions(opcionesOriginales);
        
            // Cuando cambia la seleccion de jugador 1, actualiza opciones de jugador 2
            dropdownJugador1.onValueChanged.AddListener(OnJugador1Cambio);
        
            // Controla el boton Confirmar: solo habilitalo si paises son diferentes
            botonConfirmar.interactable = false;
            dropdownJugador1.onValueChanged.AddListener(delegate { ValidarConfirmar(); });
            dropdownJugador2.onValueChanged.AddListener(delegate { ValidarConfirmar(); });
            }
        
        void OnJugador1Cambio(int index)
        {
            string paisSeleccionado = opcionesOriginales[index];
        
            // Prepara opciones para jugador 2 sin el pais que escogio jugador 1
            List<string> opcionesJugador2 = new List<string>(opcionesOriginales);
            opcionesJugador2.Remove(paisSeleccionado);
        
            // Guarda seleccion previa de jugador 2 (si existe)
            string seleccionActualJugador2 = dropdownJugador2.options[dropdownJugador2.value].text;
        
            // Actualiza las opciones de jugador 2
            dropdownJugador2.ClearOptions();
            dropdownJugador2.AddOptions(opcionesJugador2);
        
            // Si la seleccion previa sigue valida, mantenla, si no selecciona la primera opcion
            int nuevoIndex = opcionesJugador2.IndexOf(seleccionActualJugador2);
            if (nuevoIndex >= 0)
                dropdownJugador2.value = nuevoIndex;
            else
                dropdownJugador2.value = 0;
        
            dropdownJugador2.RefreshShownValue();
        
            ValidarConfirmar();
        }
        
        void ValidarConfirmar()
        {
            // Habilita el boton Confirmar solo si las selecciones son diferentes
            botonConfirmar.interactable = dropdownJugador1.value != dropdownJugador2.value;
        }
        
        public void ConfirmarSeleccion()
         {
                var seleccion = ObtenerPaisesSeleccionados();
                Debug.Log("Jugador 1 eligio: " + seleccion.Item1);
                Debug.Log("Jugador 2 eligio: " + seleccion.Item2);
        
                // Aqui continua la logica para empezar el juego con esos paises.
                // Por ejemplo: ocultar el panel y activar el mapa, cargar datos, etc.
            }
        
        
            public (string, string) ObtenerPaisesSeleccionados()
            {
                string pais1 = opcionesOriginales[dropdownJugador1.value];
                List<string> opcionesJugador2 = new List<string>(opcionesOriginales);
                opcionesJugador2.Remove(pais1);
                string pais2 = opcionesJugador2[dropdownJugador2.value];
                return (pais1, pais2);
            }
        }
        
    \end{lstlisting}

        \begin{lstlisting}[language=c, caption={BotonJugarySalir}]
        using System.Collections;
        using System.Collections.Generic;
        using UnityEngine;
        
        public class BotonJugarySalir : MonoBehaviour
        {
            // Start is called before the first frame update
            void Start()
            {
                
            }
        
            // Update is called once per frame
            void Update()
            {
                
            }
        }
    \end{lstlisting}
\end{itemize}

\subsection{Estructura del código}
\begin{description}
    \item[Assets/scripts] 
\end{description}
\begin{itemize}
    \item \colorbox{green!30}{\texttt{RecursosJugador.cs}} Lógica de recursos del jugador.
    \item \colorbox{green!30}{\texttt{classJugador.cs}} Representa a cada jugador.
    \item \colorbox{green!30}{\texttt{Departamento.cs}} Provincias del mapa.
    \item \colorbox{green!30}{\texttt{Edificios.cs}} Tipos de edificios y sus efectos.
    \item \colorbox{green!30}{\texttt{classTropas.cs}} Definición de tropas.
    \item \colorbox{green!30}{\texttt{Ejercito.cs}} Gestión de ejércitos y batallas.
    \item \colorbox{green!30}{\texttt{SistemaDeTurnos.cs}} Control de turnos.
    \item \colorbox{green!30}{\texttt{MapaControlador.cs}} Inicialización y conexiones del mapa.
    \item \colorbox{green!30}{\texttt{MenuDepartamentoUI.cs, MenuConstruccionUI.cs}}Interfaces de usuario para provincias y construcción.
    \item \colorbox{green!30}{\texttt{MenuInicialScript.cs}}Controla el menú principal y la transición a la selección de país.
    \item \colorbox{green!30}{\texttt{SeleccionPaisController.cs}} Lógica para la selección de países por los jugadores.
\end{itemize}

\subsection{Tecnologías utilizadas}
\begin{itemize}
    \item \textbf{Unity Engine}: Motor principal del juego.
    \item \textbf{C}: Lenguaje de programación.
    \item \textbf{TextMeshPro}: Para la interfaz gráfica.
    \item \textbf{Sprites y Audio}: Para la representación visual y sonora.
\end{itemize}

\subsection{Componentes clave}
\begin{itemize}
    \item \textbf{RecursosJugador}: Controla los recursos (dinero, comida, hierro, etc.), pagos y producción.
    \item \textbf{Departamento}: Nodo del grafo, representa una provincia. Gestiona edificios, tropas y producción.
    \item \textbf{Edificios}: Clases para cada tipo de edificio (Cuartel, Establo, Fábrica, Fortaleza), con sus efectos y costos.
    \item \textbf{Tropa y Ejercito}: Definen las unidades militares y su agrupación para batallas.
    \item \textbf{SistemaDeTurnos}: Controla el flujo de juego, alternando entre jugadores y procesando producción/construcción.
    \item \textbf{MapaControlador}: Inicializa el grafo de provincias y sus conexiones.
    \item \textbf{UI (MenuDepartamentoUI, MenuConstruccionUI)}: Interfaz para mostrar información y permitir acciones al usuario.
    \item \textbf{Menús principales}: 
        \begin{itemize}
            \item \textbf{MenuInicialScript}: Muestra el menú principal y gestiona la transición a la selección de país.
            \item \textbf{SeleccionPaisController}: Permite a los jugadores elegir sus países y valida la selección.
            \item \textbf{BotonJugarySalir, ScriptJugar}: Scripts para botones de navegación y acciones básicas.
        \end{itemize}
\end{itemize}


\subsection{Interacciones Entre Módulos}

\begin{itemize}
    \item \textbf{Jugador} contiene un RecursosJugador y un Ejercito
    \item \textbf{Departamento} referencia a su Dueño (Jugador) y contiene listas de Edificios y Tropas.
    \item \textbf{SistemaDeTurnos} itera sobre los jugadores y sus departamentos para procesar producción y construcciones.
    \item \textbf{UI} interactúa con los scripts de lógica para mostrar información y ejecutar acciones (construir, reclutar, atacar).
    \item \textbf{Menús} gestionan el flujo inicial del juego y la configuración de la partida.
\end{itemize}
% Comparación de Avances
\section{Comparación de Avances Semanales}

\begin{itemize}
    \item \textbf{Semana 1 - Investigación:} Se definieron las mecánicas básicas del sistema de combate. Se analizaron juegos de estrategia como referencia.
    
    \item \textbf{Semana 2 - Planeación:} Inició la creación del documento de diseño del videojuego. Se definieron los objetivos generales del proyecto y su estructura inicial. \textbf{Problema:} Se está tardando más de lo esperado en consolidar toda la información en el documento.

    \item \textbf{Semana 3 - Diseño del mapa:} Se completó el boceto del mapa con divisiones políticas por departamentos. El diseño se basó en referentes históricos y geográficos.

    \item \textbf{Semana 4 - Backend:} Se desarrollaron clases clave para el funcionamiento del juego (provincias, jugadores, recursos). Se inició la conexión con una base de datos local para gestionar información persistente. \textbf{Problema:} Algunas clases aún no están completamente integradas entre sí.

    \item \textbf{Semana 5 - Pantalla principal:} Se completó la pantalla inicial del juego con navegación básica. La estética visual fue alineada con la temática histórica del juego.

    \item \textbf{Semana 6 - Menús e implementación de turnos:} Se crearon menús interactivos para cada provincia. Se implementó el sistema de turnos como base del flujo del juego.

    \item \textbf{Semana 7 - Sistema de combate y recursos:} Se ajustaron las reglas de combate para facilitar la jugabilidad. Se finalizó el menú principal con diseño funcional. Se añadió la lógica para asignar recursos a cada departamento.
\end{itemize}

\end{document}

